\documentclass[a4paper, 11pt]{report}

\usepackage{palatino}
\usepackage{mathpazo}

\usepackage[margin=1in]{geometry}
\usepackage{microtype}

\usepackage{mathtools}
\mathtoolsset{centercolon}

\usepackage[]{hyperref} 
\usepackage[]{cleveref} 
\usepackage[]{enumerate} 
\usepackage[]{makeidx} 
\usepackage[]{amsthm} 
\usepackage[]{thmtools} 
\renewcommand{\listtheoremname}{List of Exercises}
\makeindex

\theoremstyle{plain}
\newtheorem{thm}{Theorem}[section]
\theoremstyle{definition}
\newtheorem{defn}[thm]{Definition}
\newtheorem{exmp}[thm]{Example}
\newtheorem{exrc}[thm]{Exercise}
\newtheorem{prbl}[thm]{Problem}
\newtheorem*{sltn}{Solution}

\newcommand{\impl}{\Longrightarrow}
\newcommand{\eqvl}{\Longleftrightarrow} 
\renewcommand{\neg}{{\sim}} % rewriting the \sim command as an unary negation
\title{
  Notes and exercises in MAT1140
}
\author{
  Ivar Stangeby
}

\begin{document}

\maketitle

\abstract
This document is a series of notes and commentaries related to the course
MAT1140 - Structures and Arguments lectured at the University of Oslo (UiO).
This is my attempt to organize and structure my work in this course, as well as
giving me an easy way of referring back to the course material. The document is
highly personal, but if someone can get any use from it, then by all means - go
ahead! 
\tableofcontents

\chapter{Logic}
\label{cha:logic}

\section{True or False}
\label{sec:true_or_false}

We first want to examine some thought experiments. Examine each one and figure
out what the statement means, and whether it is true or false.

\begin{enumerate}
  \item The points $(-1, 1), (2, -1)$ and $(3, 0)$ lie on a line.

    We know from linear algebra, that if these points are to lie on the
    same line, they all have to satisfy the same linear equation $ax + b$.
    These do not, so the statement is false.

  \item If $x$ is an integer, then $x^2 \geq x$.

    This is an implication, on the form "if $A$, then $B$". We are
    interested in the world where $x$ is an integer, and want to examine
    whether this means that $x^2 \geq x$ must hold. We have three cases, $x
    > 0, x = 0$ and $x < 0$.  It holds for each case, so the statement is
    true. To prove this rigorously, we should use axiomatic set theory and
    the trichotomy of the integers.

  \item If $x$ is an integer, then $x ^ 3 \geq x$.

    This is a statement on the same form as above. This one however, is
    false, and we can show this just by finding a counter-example. We know
    that any negative number raised to an odd power is also negative.
    Hence, if we let $x < 0$ and $|x| > 1$ we must have that $x ^ 3 < x$.
    Thus, the statement is false.  

  \item For all real numbers $x$, $x^3 = x$.

    This again, is an implication, if one restates the statement.  If $x$
    is a real number, then $x ^ 3 = x$. Knowing that all integers are real
    numbers we can use the knowledge from the previous one to instantly see
    that this must be false. Just pick any real number other than $0$ and
    $1$ really.

  \item There exists a real number $x$ such that $x ^ 3 = x$.

    This statement is in some sense a weaker version of the statement
    above. It only asserts the existence of at least one. Hence, all we
    have to do is find a real number that satisfies the equation. If we let
    $x = 1$, then we have $1 ^ 3 = 1$, hence the statement is true.

  \item $\sqrt{2}$ is an irrational number.

    To show this statement, we must remind ourselves what an irrational
    number is.  An irrational number is a number on the form $a / b$ where
    $a, b \in \mathbb{Z}$ and $b \neq 0$.  The statements essentially
    asserts that there are no numbers on the form $x = a / b$ such that $x
    ^ 2 = 2$.  We can assume for contradiction that there does exist some
    rational number $x$ such that $x ^ 2 = 2$. We can then write $x =
    \frac{a}{b}$, and we can also assume that the numbers are in their
    lowest terms. That any common factors have been canceled out.  If we
    square $x$, we get $x ^ 2 = \frac{a^2}{b^2}$. Remember, that we assumed
    that the square of $x$ was equal to $2$, so we have that $2 =
    \frac{a^2}{b^2}$. Multiplying through by $b^2$ we get that $2b^2 =
    a^2$, hence $a^2$ must be an even number. But, in that case $a$ must be
    an even number, since the square of an even number is
    even.\footnote{This is shown in no. 8.} Thus, we can write $a$ on the
    form $2n$ for some integer $n$. If we substitute this into our previous
    equation we get that $2b^2 = 4n^2$, and the $2$'s cancel and we get
    $b^2 = 2n^2$, hence $b$ must be an even number. But this means that
    both $b$ and $a$ have the common factor of $2$, and since we assumed
    that $a$ and $b$ had no common factors this is the contradiction we
    wanted. Hence, $\sqrt{2}$ is an irrational number.

  \item If $x + y$ is odd and $y + z$ is odd, then $x + z$ is odd.

    A number is odd if there exists some integer $n$ such that the number
    can be written on the form $2n + 1$. We therefore let $n, m \in
    \mathbb{Z}$ such that $x + y = 2n + 1$ and $y + z = 2m + 1$. Solving
    these equations for $x$ and $z$ respectively, we get that $x = 2n + 1 -
    y$ and $z = 2m + 1 - y$. Adding we see that $x + z = 2(n + m + 1 - y)$.
    Hence $x + z$ is an \textit{even} number, not an odd one.

  \item If $x$ is an even integer, then $x^2$ is an even integer. 

    Since $x$ is an even integer, we can write it on the form $x = 2n$ for
    some integer $n$. If we square $x$, we see that $x^2 = 4n^2$, which we
    can rewrite as $2(2n^2)$. Hence $x^2$ is an even integer. This fact was
    used in the proof for the irrationality of the square root of two
    above.

  \item Every positive integer is the sum of distinct powers of two.

    This one is tricky to solve, and I have no idea whether it is true or
    not.  By testing some cases one can easily see that most numbers follow
    this, and I have not found any counter-example.

  \item Every positive integer is the sum of distinct powers of three.

    Same as above, no idea.

  \item If $x$ is an integer, then $x$ is even or $x$ is odd.

    An even number is on the form $2n$ for some integer $n$, and hence,
    must be an integer.  The same goes for odd numbers on the form $2n +
    1$. Hence, if $x$ is even it must be an integer, and if $x$ is odd it
    must be an integer.  It now remains to show that if $x$ is an integer,
    then it cannot be neither even nor odd. If we assume that $x$ is an
    integer, but is neither even nor odd there must in some sense be a
    decimal part to $x$. And in that case it cannot be an integer.

  \item If $x$ is an integer, then $x$ cannot be both even and odd.

    If we assume $x$ to be an integer that can be written on the form $x =
    2n$ and $x = 2m + 1$ for some integers $n, m \in \mathbb{Z}$, then we
    can add the two expressions together to achieve
    \begin{equation}
      \notag
      2x = 2n + 2m + 1 = 2(n + m) + 1.
    \end{equation}
    If we solve this for $x$, we get that $x = 2(n+m) + \frac{1}{2}$. Hence
    $x$ is not an integer, which contradicts our initial assumption.  This
    concludes the proof.

  \item Every even integer greater than $2$ can be expressed as the sum of two prime numbers.

    This is one of the unsolved problems in mathematics. I do not know how
    to solve this one, but later in this chapter I will attempt to rewrite
    the statement into equivalent ones to get different perspectives on the
    problem.

  \item There are infinitely many prime numbers.

    A prime number is a whole number that is only divisible by $1$ and
    itself. If we assume that there are a finite number of prime numbers
    $p_0, p_1, \ldots, p_n$ then we know that these have no common factors.
    We can now construct a new prime number, called $q$ like this:
    \begin{equation}
      \notag
      q = p_0\cdot p_1 \cdot p_2 \cdot \ldots \cdot p_n + 1
    \end{equation}

    This new number $q$ cannot possibly have any common factors with the
    other prime numbers, hence it is divisible only by itself and one.  We
    can repeat this process indefinitely to generate an infinite amount of
    prime numbers. Hence the statement is true.

  \item For any positive real number $x$ there exists a positive real number $y$ such that $y^2 = x$.

    This statement essentially asserts that given any real number $x$, we
    can always find another real number $y$ such that if you square this
    number $y$ you get $x$. Not quite sure how to prove this.

  \item Given three distinct points in space, there is one and only one plane passing through them.

    This is fairly obvious, and this is a typical uniqueness-proof in
    mathematics. By assuming that there are two distinct planes passing
    through the same points, one can show that these two planes must
    actually be equal.

\end{enumerate} 

I am now to divide my answers into four categories, namely the following:
\begin{enumerate}[a)]
  \item I am confident that the justification I gave is conclusive.

    1, 3, 4, 5, 6, 7, 8, 12, 14
  \item I am not confident that the justification I gave is conclusive.

    2
  \item I am confident that the justification I gave is \textit{not} conclusive. (If you gave no justification at all, your answer falls
    into this category.)

    9, 11, 13, 15, 16 
  \item I could not decide whether the statement was true or false.

    None
\end{enumerate}

\section{Statements and Predicates}
\label{sec:statements_and_predicates}

A \textbf{statement}\index{statement} is a sentence that is either true or false, but not ambiguous.


\begin{exrc}
  Give some examples of sentences that are statements and some examples of sentences that are not statements.
\end{exrc}

\begin{sltn}
\begin{enumerate}
  \item \textit{World War 2 ended in 1945} is a statement.
  \item \textit{The university campus was tranquil during the summer} is not a statement. What university are we talking about?
  \item \textit{The Eiffel tower is made of metal} is a statement.
  \item \textit{The building is made of bricks} is not a statement. Which building?
\end{enumerate}
\end{sltn}

A sentence with one or more free variables in it that becomes a statement
when the free variables takes on a particular value is called a
\textbf{predicate}\index{predicate}.

\begin{exrc}
  Give examples of mathematical predicates that have two and three free variables.
\end{exrc}

\begin{sltn}
\begin{enumerate}
  \item \textit{He ate bacon and eggs for breakfast} is a predicate with one free variable, "He".
  \item \textit{He said to her that he was leaving} is a predicate with two free variables, "He" and "She".
  \item $x ^ 2 + 3y = 12z$ is a predicate with three free variables, namely $x, y$ and $z$.
\end{enumerate}
\end{sltn}

\section{Quantification}

We can turn a predicate into a statement by substituting particular values
for its free variables. There are at least two other ways in  which
predicates can be used to build statements. We make a claim about which
values of the free variable turn the predicate into a true statement.

The phrases \textbf{for all}\index{for all} and \textbf{there
exists}\index{there exists} are called
\textbf{quantifiers}\index{quantifiers}, and the process of using
quantifiers to make statements out of predicates is called
\textbf{quantification}\index{quantification}.


\begin{exrc}
  Suppose we understand the free variable $z$ to refer to fish.
\end{exrc}

\begin{sltn}
If we let $A(z)$ stand for the predicate \textit{$z$ lives in the sea},
then the statement "for all $z$, $A(z)$" is true.

If we let $B(z)$ stand for the predicate \textit{$z$ is blue}, then the statement
"for all $z, B(z)$" is false but "there exists $z$ such that $B(z)$" is true.
\end{sltn}

"For all" is called the \textbf{universal quantifier}\index{universal
quantifier} and has the symbol $\forall$\index{$\forall$}.  "There exists \ldots such that"
is called the \textbf{existential quantifier}\index{existential quantifier}
and has the symbol $\exists$.\index{$\exists$}

You might find it curious that a sentence can contain a variable, as quantified
statements do, and yet be a statement. We call the variables in a statement
with quantifiers \textbf{bound variables}\index{bound variables}. This
essentially means that when the variable is bound there is no ambiguity.
Note that the order of quantification matters greatly.

\begin{exrc}
  Consider the following two statements.
  \begin{enumerate}
    \item There exists $x$ and there exists $y$ such that $y^2 = x$.
    \item There exists $y$ and there exists $x$ such that $y^2 = x$.
  \end{enumerate}
  Did quantifying over $y$ and then $x$ rather than the other way around change
  the meaning of the statement? What if the quantifiers had been \textit{for
  all} instead of \textit{there exists}?
\end{exrc}

\begin{sltn}
It seems like the order of quantifiers do not matter as long as the
quantifiers are the same. In other words, having two "for all" quantifiers it
does not matter what order and similarly for two "there exists" quantifiers.
However, if we have one "for all" and one "there exists" changing the order
definitely changes the meaning of the statement. Examine the statements
\begin{enumerate}
  \item There exists $x$ such that for all $y$, $y^2 = x$.
  \item For all $x$ there exists $y$ such that $y^2 = x$.
\end{enumerate}
The first one asserts the existence of one single $x$ such that the equation
holds for all $y$, which is obviously false. The second one however asserts the
fact that no matter what number $x$ you pick you can always find a number $y$
that has $x$ as its square.
\end{sltn}

\begin{exrc}
  Consider the predicate about integers "$x = 2y$". There are six distinct ways
  of quantifying this statement. Find all six and determine the truth value.
\end{exrc}

\begin{sltn}
From the previous exercise, we can change the order of quantifiers without
altering the meaning of the statement. Thus, we have six distinct ways of
quantifying the statements.

\begin{enumerate}
  \item $\forall x \exists y$ such that $x = 2y$. Chose $x = 1$, this is false.
  \item $\exists x \forall y$ such that $x = 2y$. False.
  \item $\exists x \exists y$ such that $x = 2y$. True, $x = 2$ and $y = 1$.
  \item $\forall x \forall y$ such that $x = 2y$. False.
  \item $\forall y \exists x$ such that $x = 2y$. True.
  \item $\exists y \forall x$ such that $x = 2y$. False.
\end{enumerate}
\end{sltn}

\section{Mathematical Statements}
\label{sec:mathematical_statements}

The vast majority of mathematical statements can be written in the form "If
$A$, then $B$" where $A$ and $B$ are predicates. The question now is, if $A$
and $B$ are predicates then surely "If $A$, then $B$" should be a predicate as
well, not a statement. However, it is standard practice to assume universal
quantification over the variables. In other words, the statement is to be read
as "For all $x$, if $A(x)$, then $B(x)$".

\begin{defn}
  A statement on the form "If $A$, then $B$", where $A$ and $B$ are statements
  or predicates is called an \textbf{implication}\index{implication}. $A$ is
  called the \textbf{hypothesis}\index{hypothesis} of the statement and $B$ is
  called the \textbf{conclusion}\index{conclusion}.
\end{defn}

\addtocounter{thm}{1}

\begin{exrc}
  Identify the hypotheses and conclusions in each of the implications given in
  the previous example. (pg. 14)
\end{exrc}

\begin{sltn}
\begin{enumerate}
  \item Hypothesis: $x + y$ is odd and $y + z$
        Conclusion: $x + z$ is odd
  \item Hypothesis: $x$ is an integer
        Conclusion: $x$ either even or odd, but not both
  \item Hypothesis: $x^2 < 17$
        Conclusion: $x$ is a positive real number.
  \item Hypothesis: $x$ is an integer
        Conclusion: $x^2 \geq 2$
  \item Hypothesis: $f$ is a polynomial of odd degree
        Conclusion: $f$ has at least one real root
\end{enumerate}
\end{sltn}

Remember that most mathematical statements can be rephrased as implications!

\section{Mathematical Implication}
\label{sec:mathematical_implication}

We first start by talking about what it means for an implication to be true.
\begin{exmp}
  If $x$ is an integer, then $x^2 \geq 2$.  
\end{exmp}
\begin{proof}
  If $x = 0$, then $x^2 = x$, so certainly $x^2 \geq x$. The same is true if $x
  = 1$. If $x \geq 1$, then $x^2 > 1 \cdot x = x$ If $x < 0$ then $x^2 > 0 >
  x$. This accounts for all integer values of $x$.
\end{proof}

Here we studied all the values of the variable $x$ that mate the hypothesis
true. We then showed that in those cases the conclusion also is true. We did
not care for the values of $x$ that were not integers because they are not relevant for our case.

\begin{exrc}
  We now consider the statement "If $x$ is an integer, then $x^3 \geq x$."
  \begin{enumerate}
    \item Show that "If $x$ is an integer, then $x^3 \geq x$" is false.  \item
      Thinking in terms of hypotheses and conclusions, explain what you did to
      show that the statement is false.
  \end{enumerate}
\end{exrc}

\begin{sltn}
To show that this statement is false, we simply have to find one
counter-example where we have an integer, but its cube is less than itself.  If
we for example let $x = -2$ we have that $x^3 = -8$ hence the statement is
clearly false.

In terms of hypotheses and conclusions, we only considered the values for $x$
that make the hypothesis true. Namely the integers. Then proceeded to find a
value for $x$ that satisfied the hypothesis yet did not satisfy the conclusion.
Hence a contradiction was found and the statement was proved to be false.
\end{sltn}

A value of $x$ that makes the hypothesis $A$ true and the conclusion $B$ false
is called a \textbf{counterexample}\index{counterexample}.

\begin{exrc}
  Occasionally you will see "If $A$, then $B$" written as "$A$ is sufficient
  for $B$" or "$B$ is necessary for $A$"or "$B$, if $A$" or "$A$ only if $B$".
  Explain why it is sensible to say that each of these means the same thing.
\end{exrc}
\begin{sltn}
"$A$ is sufficient for $B$" means that if we have that $A$ is true, then $B$
also must be true because that $A$ is true is sufficient. This is equivalent to
the implication. "$B$ is necessary for $A$" means that if we have $A$ then we
also must have $B$. They essentially all mean the same thing.\footnote{I have
no idea how to properly explain this}
\end{sltn}

An implication in which the hypothesis is false is often said to be
\textbf{vacously true}\index{vacuously true}. We define the implication to be
true when the hypothesis is false because values that make the hypothesis false
can never generate a counterexample.

\section{Compound Statements and Truth Tables}
\label{sec:compund_statements_and_truth_tables}

Symbolically we write "If $A$, then $B$" as $A \impl B$\index{$\impl$}, which is
read "\textbf{A implies B}". 

\begin{table}[h!]
  \centering
  \caption{Truth table for implication}
  \label{tab:implication}
  \begin{tabular}{ccc}
    \hline
    A & B & A $\impl$ B \\
    \hline
    T & T & T \\
    T & F & F \\
    F & T & T \\
    F & F & T \\
    \hline
  \end{tabular}
\end{table}

The above table is an example of a \textbf{truth table}\index{truth table}.
In addition to implication we can combine statements $A$ and $B$ in a number of ways.
\begin{enumerate}
  \item "A and B" is called the \textbf{conjunction}\index{conjunction}\index{$\land$} of A and B. This is denoted $A \land B$.
  \item "A or B" is called the \textbf{disjunction}\index{disjunction}\index{$\lor$} of A and B. We denote it by $A \lor B$.
  \item "Not A" is called the \textbf{negation}\index{negation}\index{$\neg$} of A. We denote it $\neg A$.
  \item "A if and only if B" is called the
    \textbf{equivalence}\index{equivalence}\index{$\eqvl$} of A and B. We
    denote it by $A \eqvl B$. ("If and only if" is often abbreviated
    \textbf{iff}\index{iff}.)
\end{enumerate}

\begin{defn}
  Suppose that A and B are statements. The following truth table gives the
  truth values of all the previously mentioned compound statements.
  \begin{table}[h!]
    \centering
    \caption{Compound statements}
    \label{tab:comp_statements}
    \begin{tabular}{ccccccc}
    \hline 
    {} & {} & A implies B & A and B & A or B & not A & A iff B \\
     A & B & $A \impl B$ & $A \land B$ & $A \lor B$ & $\neg A$ & $A \eqvl B$ \\
     T& T& T& T& T& F& T \\
     T& F& F& F& T& F& F \\
     F& T& T& F& T& T& F \\
     F& F& T& F& T& T& T \\
    \hline
    \end{tabular}
  \end{table}
\end{defn}

\begin{exrc}
  Examine \cref{tab:comp_statements}. Given the colloquial meaning of terms
  "and", "or", "not" and "equivalent" explain why the truth values in the table
  make sense.
\end{exrc}
\begin{sltn}
\begin{enumerate}
  \item "and"
    When we say that we want A \textit{and} B we want to have both at the same
    time. Hence if either A or B is false then we do not have both and hence
    the statement is false. This coincides with the truth table.
  \item "or"
    When we say "this or that" we mean "either this, or that, or both"
    therefore we only require at least one of A or B to be true for the
    statement "A or B" to be true.
  \item "not"
    Not is simply the negation. Hence if we have something, then the negation
    of that is not having something.
  \item "equivalent"
    Two statements are equivalent if one can swap one for the other without
    losing any meaning. In other words, we must have both A and B true or A and
    B false for the equivalence to be true.
\end{enumerate}
\end{sltn}

\begin{exmp}
  Given that $A$ and $B$ are statements we can look at the compound statements
  \begin{enumerate}
    \item $B \land \neg B$:
      If we construct the truth table for this compound statement we see that
      the statement is always false. A compound statement that is always false
      is regardless of the truth value of the simpler statements involved is
      called a \textbf{contradiction}\index{contradiction}. 
    \item $(A \land \neg B) \eqvl \neg (A \impl B)$:
      This compound statement is true no matter what the truth value of the
      simpler statements involved is called a
      \textbf{tautology}\index{tautology}.
  \end{enumerate}
\end{exmp}

\pagebreak
\begin{exrc}
  Verify that
  \begin{equation}
    \notag
    (A \impl (B \lor C)) \eqvl ((A \land \neg B) \impl C)
  \end{equation}
  is a tautology.
\end{exrc}
\begin{sltn}
  \begin{table}[h!]
    \centering
    \begin{tabular}{cccccccc}
     \hline
     A& B & C & $B \lor C$ & $\neg B$ & $A \land \neg B$ & $A \impl (B \lor C)$ & $(A \land \neg B)\impl C$ \\
     \hline
     T& T & T & T & F & F & T & T\\ 
     T& T & F & T & F & F & T & T\\ 
     T& F & T & T & T & T & T & T\\ 
     T& F & F & F & T & T & F & F\\ 
     F& T & T & T & F & F & T & T\\ 
     F& T & F & T & F & F & T & T\\ 
     F& F & T & T & T & F & T & T\\ 
     F& F & F & F & T & F & T & T\\ 
     \hline
    \end{tabular}
  \end{table}
We here see that the statements $(A \impl (B \lor C))$ and $((A \land \neg B)
\impl C)$ have exactly the same truth values. We know that an equivalence only
hold if the two statements have the same truth values for all possible values
of the variables. Hence the equivalence is true no matter what, and we call it
a tautology.
\end{sltn}

\section{Learning from Truth Tables}
\label{sec:learning_from_truth_tables}

\begin{exrc}
  Consider the following statements.
  \begin{enumerate}
    \item $(A \impl (B \land C)) \impl (A \impl B)$.
    \item $(A \land (A \impl B)) \impl B$.
    \item $((A \impl B) \land (B \impl C)) \impl (A \impl C)$.
  \end{enumerate}
  Each of these statements are tautologies. First verify using truth tables
  that the statements are tautological. Then figure out what logical principles
  each statement embodies and what they tell us about proving theorems.\footnote{I am going to skip creating the truth tables as they take some time.}
\end{exrc}
\begin{sltn}
\begin{enumerate}
    \item $(A \impl (B \land C)) \impl (A \impl B)$.
      This statement tells us that if $A$ implies that $B$ and $C$ are true,
      then that implies that $A$ implies that $B$ is true. This means, in terms
      of proving theorems that we can show that the hypothesis implies a
      stronger conclusion ($B \land C$) than the one we want to show, and thus
      deduce that it must also hold for the weaker conclusion which we wanted
      to show ($B$).
    \item $(A \land (A \impl B)) \impl B$.
      If we know that $A \impl B$ is true, and that $A$ is true, then we can
      conclude with $B$ also being true.
    \item $((A \impl B) \land (B \impl C)) \impl (A \impl C)$.
      This statement tells us that implication has the transitive property. If
      both $A$ implies $B$ and $B$ implies $C$, then it must follow that $A$
      implies $C$. In terms of proving theorems, if we want to show that $A$
      implies $C$, then and we know that $B$ implies $C$ we can instead chose
      to prove that $A$ implies $B$. 
\end{enumerate}
\end{sltn} 
\begin{defn}
  The implication $B \impl A$ is called the \textbf{converse}\index{converse} of $A \impl B$.
\end{defn}

\pagebreak
\begin{exrc}
  Construct a truth table to show that it is possible for $A \impl B$ to be
  true while its converse $B \impl A$ is false, and vice versa.
\end{exrc}
\begin{sltn}
\begin{table}[h!]
  \centering
  \begin{tabular}{cccc}
    \hline
    A & B & A $\impl$ B & B $\impl$ A \\
    \hline
    T & T & T& T\\
    T & F & F& T\\
    F & T & T& F\\
    F & F & T& T\\
    \hline
  \end{tabular}
\end{table}
We here see that the implication and its converse are only equal in the two
cases where $A$ and $B$ have the same truth value. Hence, we cannot simply flip
the implication sign without changing the meaning of the statement.  Thus, we
have to treat a statement and its converse as two distinct mathematical claims.
\end{sltn}
\stepcounter{thm}
\begin{exrc}
  Find an example of a true statement whose converse is false and one whose converse is true. 
\end{exrc}
\begin{sltn}
\begin{enumerate}
  \item "If he lives in Oslo, then he lives in Norway" is true but has a false converse.
  \item "If $x^3 = 1$, then $x = 1$" is true, and has a true converse.
\end{enumerate}
\end{sltn}

Let us consider the truth table for $A \eqvl B$:
\begin{table}[h!]
  \centering
  \caption{Equivalence}
  \label{tab:equivalence}
  \begin{tabular}{ccc}
  \hline
   A&  B& A $\eqvl$ B\\
   \hline
   T&  T& T\\
   T&  F& F\\
   F&  T& F\\
   F&  F& T\\
   \hline
  \end{tabular}
\end{table}

We see that $A$ and $B$ must have the same truth value for the equivalence to
be true.  If $A \eqvl B$ is true, then we say that $A$ and $B$ are
\textbf{equivalent}\index{equivalent}. If we manage to prove $A$ we know that
$B$ is true as well and conversely.

Equivalence of statements are important because when proving a statement we can instead chose to prove an equivalent statement. Consider a statement on the form "If $A$, then $B$ or $C$", we can instead chose to prove either
\begin{enumerate}
  \item If $A$ and not $B$, then $C$.
  \item If $A$ and not $C$, then $B$.
\end{enumerate}

\begin{exrc}
  Show by constructing a truth table that 
  \begin{equation}
    \notag
    (A \eqvl B) \eqvl ((A \impl B) \land (B \impl A))
  \end{equation}
  is a tautology.
\end{exrc}
\begin{sltn}
\begin{table}[h]
  \centering
  \begin{tabular}{ccccc}
    \hline
    A & B & A $\eqvl$ B & A ($\impl$ B) $\land$ (B $\impl$ A))\\
    \hline
    T& T& T& T\\
    T& F& F& F\\
    F& T& F& F\\
    F& F& T& T\\
    \hline
  \end{tabular}
\end{table}
\end{sltn}

Hence, we can when proving equivalence of statements prove the one sided
implication and its converse instead. This is an important thing to remember.

\begin{exrc}
  Construct a truth table for the following four statements:\\
  \begin{tabular}{cccc}
    $\neg(A\land B)$& $\neg A\land \neg B$ & $\neg(A\lor B)$ & $\neg A \lor \neg B$
  \end{tabular} 
\end{exrc}
\begin{sltn}
  \begin{table}[h!]
    \centering
    \begin{tabular}{cccccc}
    \hline
    A&  B&  $\neg(A\land B)$& $\neg A\land \neg B$ & $\neg(A\lor B)$ & $\neg A \lor \neg B$ \\
    \hline
   T&  T&  F&  F& F & F  \\
   T&  F&  T&  F& F & T \\
   F&  T&  T&  F& F & T \\
   F&  F&  T&  T& T & T \\
    \hline
    \end{tabular}
  \end{table}
\end{sltn}
  We can from this see that the negation of a conjunction is the disjunction of
  the negations and the negation of a disjunction is the conjunction of the
  negations. These are what we call \textbf{De Morgan's laws}.\index{De
  Morgan's laws}

  \section{Negating Statements}
  \label{sec:negating_statements}
  
  Generally it is more useful to have a statement phrased positively. We can
  say "It is not true that $A$" but it is more useful to say what \textit{is} true, rather than what is not. For example, we typically rephrase the
  statement "$x$ is not even" as "$x$ is odd".

  \begin{exrc}
    Rephrase the statement "$x$ is not greater than 7" in positive terms.
  \end{exrc}
  
  \begin{sltn}
    If $x$ is not greater than 7 then it has to be less than or equal to 7.
    Hence the statement in positive terms is "$x$ is less than or equal to 7".
  \end{sltn}

  Sometimes it is impossible to rephrase a negative statement as a positive
  statement. In that case, the negative statement will have to do.

  \begin{exrc}
    Think colloquially about the meaning of AND and AND/OR. Explain why it
    makes sense for the negation of $A \lor B$ to be $\neg A \land \neg B$ and
    for the negation of $A \land B$ to be $\neg A \lor \neg B$.
  \end{exrc}
  \begin{sltn}
    What does it mean for $A \lor B$ to be false? Well, it is true if either
    $A$  $B$ or both are true, hence we need both $A$ and $B$ to be false for
    the negation to be true.  What does it mean for $A \land B$ to be false? It
    is true only if $A$ and $B$ are true, hence we need $A$ or $B$ or both to
    be false for the negation to be true.
  \end{sltn}

  \begin{exrc}
    Negate the following statements and if possible restate the statement in positive terms.
    \begin{enumerate}
      \item $x + y$ is even and $y + z$ is even.($x, y, z$ are fixed integers)
      \item $x > 0$ and $x$ is rational. ($x$ is a fixed real number)
      \item Either $l$ is parallel to $m$ or $l$ and $m$ are the same line.($l$ and $m$ are fixed lines in $\mathbb{R^2}$)
      \item The roots of this polynomial are either all real or all complex. 
    \end{enumerate}
  \end{exrc}
  
  \begin{sltn}
    Since the variables are fixed we do not have to worry about the assumed
    universal quantification.
    \begin{enumerate}
      \item This is a conjunction. By De Morgans laws the negation is "$x + y$
        is odd or $y + z$ is odd".
      \item This is also a conjunction. The negation is "$x \leq 0$ or $x$ is
        irrational."
      \item This is a disjunction.  "$l$ is not parallel to $m$ and $l$ and $m$
        are not the same line". This can also be rephrased in positive terms as
        "$l$ intersects $m$ and $l$ and $m$ are distinct lines."
      \item This is a conjunction, so the negation becomes "There exist a root
        of this polynomial that is not real nor complex. Remember we have the
        assumed universal quantifier, hence we get an "There exist" in the
        negation.
    \end{enumerate}
  \end{sltn}

  \begin{exrc}
    Using similar reasoning, find the negation of the statement "There exists a fast snail."
  \end{exrc}
  \begin{sltn}
    What does it mean for this statement to be false? In that case we must have
    no fast snails. Hence we get the negation "All snails are slow".
  \end{sltn}
  
  \begin{exrc}
    Negate the following statements and rephrase them in positive terms.
    \begin{enumerate}
      \item There exists a line in the plane passing through the points $(-1, 1), (2, -1)$, and $(3, 0)$.
      \item There exists an odd prime number.
      \item For all real numbers $x, x^3 = x$.
      \item Every positive integer is the sum of distinct powers of three.
      \item For all positive real numbers $x$ there exists a real number $y$
        such that $y^2 = x$
      \item There exists a positive real number $y$ such that for all real
        numbers $x, y^2 = x$.
    \end{enumerate}
  \end{exrc}
  \begin{sltn} We negate each statement:
    \begin{enumerate} 
      \item All lines in the plane does not pass through the points $(-1, 1), (2, -1)$, and $(3, 0)$.
      \item All prime numbers are even.
      \item There exists a real number $x$ such that $x^3 \neq x$.
      \item There exists a positive integer that is not the sum of distinct powers of three.
      \item There exists a real number $x$ such that for all real numbers $y$, $y^2 \neq x$.
      \item For all positive real numbers $y$ there exists a real number $x$
        such that $y^2 \neq x$.
    \end{enumerate} 
  \end{sltn}

  \stepcounter{thm}
  We now look at a more complex example. Consider the statement "All Martians
  are short and bald, or my name isn't Darth Vader".  We can negate this by
  considering the various sub-statements.  The negation becomes "Either some
  Martian is tall or some Martian has hair, and my name is Darth Vader."

  We can now try our hands at the following problem
  \begin{prbl}
    Try to negate
    \begin{quote}
      \textit{You can fool some of the people all of the time, and some of the
      people none of the time, but you cannot fool all of the people all of the
    time.}
    \end{quote}
  \end{prbl}

  \begin{sltn}
    We need to decompose this statement into the various sub-statements.  Let
    $A(x)$ = You can fool $x$ all of the time" and $B(x)$ = "You can fool $x$
    none of the time". Our statement then become, having $x$ refer to a person.
    There exists an $x$ such that $A(x)$ and there exist an $x$ such that
    $B(x)$, but not all $x$ can be fooled all of the time.\footnote{This is not
    finished, having problems creating the sub-statements I need in order to
  show this. I'll revisit this later.}
  \end{sltn}
 
  \pagebreak
  \begin{exrc}
    Show by constructing a truth table the equivalence of the two statements:
    \begin{quote}
      $\neg (A \impl B)$ and $A \land \neg B$.
    \end{quote}
  \end{exrc}
  \begin{sltn}
    \begin{table}[h!]
      \centering
      \caption{Negating an implication}
      \begin{tabular}{cccc}
        \hline
        A & B & $ \neg(A \impl B)$ & $A \land \neg B$\\
        \hline
        T& T& F& F\\
        T& F& T& T\\
        F& T& F& F\\
        F& F& F& F\\
        \hline
      \end{tabular}
    \end{table}
    This makes sense, since what does it mean for an implication to be false?
    Namely to have a value that makes the hypothesis true but the conclusion
    false.
  \end{sltn}
  \begin{exrc}
    Negate the statement "If $x^2 > 14$, then $x < 10$."
  \end{exrc}
  \begin{sltn}
    This is an implication, so we use what we just learned from the previous
    exercise. We need the hypothesis to be true and the conclusion to be false.
    Hence the negation becomes "$x ^2 > 14$ and $x > 10$". Actually, we have to
    also negate the assumed universal quantification in the implication. So the
    statement \textit{really} becomes "There exists an $x$ such that $x ^ 2 >
    14$ and $x > 10$.
  \end{sltn}

  \begin{exrc}
    Negate the following statements and rephrase them in positive terms.
    \begin{enumerate}
      \item If $x$ is an odd integer, then $x^2$ is an even integer.
      \item If $x + y$ is odd and $y + z$ is odd, then $x + z$ is odd.
      \item If $f$ is a continuous function, then $f$ is a differentiable
        function.
      \item If $f$ is a polynomial, then $f$ has at least one real root.
    \end{enumerate}
  \end{exrc}
  \begin{sltn}
    Remembering the assumed universal quantification we negate each
    statement.\footnote{Being fairly verbose here, just to emphasize the
    quantifiers.}
    \begin{enumerate}
      \item There exists $x$ such that $x$ is an odd integer and $x^2$ is an
        odd integer.
      \item There exists an $x$ and there exists a $y$ and there exists a $z$
        such that $x + y$ is even or $y + z$ is even and $x + z$ is even.
      \item There exists an $f$ such that $f$ is a continuous function and $f$
        is not differentiable.
      \item There exists an $f$ such that $f$ is a polynomial and $f$ has no
        real roots.
    \end{enumerate}
  \end{sltn}

  \begin{exrc}
    Let $f$ be the function $ f : \mathbb{R} \rightarrow \mathbb{R}$. Negate
    the following statement:
    \begin{quote}
      \textit{For all positive real numbers $r$, there exists a positive real
      number $s$ such that if the distance from $y$ to $3$ is less than $s$,
    then the distance from $f(y)$ to $7$ is less than $r$.}
    \end{quote}
  \end{exrc}
  \begin{sltn}
  I am going to rewrite this in mathematical notation, just to make the negation easier.
  We have:
  \begin{equation}
    \notag
    \forall r > 0, \exists s > 0 \text{ s.t } |y - 3| < s \impl |f(y) - 7| < r.
  \end{equation}
  We here see that we have quantified over all variables except $y$, hence we
  have to assume universal quantification over $y$. This gives us
  \begin{equation}
    \notag
    \forall r > 0, \exists s > 0 \text{ s.t }\forall y,  |y - 3| < s \impl |f(y) - 7| < r.
  \end{equation}

  We can now consider the sub-statements involved. 
  The statement is on the form $\forall r \exists s \text{ s.t } (A(s, y) \impl B(y, r))$,
  which in turn can be written as $\forall r \exists s$ s.t $P(s, r, y)$.
  We start negating:
  \begin{enumerate}[i.]
    \item $\exists r \forall s \exists y (\neg P(s, r, y))$.
    \item $\exists r \forall s \exists y (A(s, y) \land \neg B(r, r))$.
    \item $\exists r \forall s \exists y (|y - 3| < s \land |f(y) - 7| \geq r)$.
  \end{enumerate}
  Thus the negated statement is
  \begin{quote}
    \textit{There exists a positive real number $r$ such that for all positive
    real numbers $s$ there exists a $y$ such that the distance from $y$ to $3$
  is less than $s$ but the distance from $f(y)$ to 7 is greater than or equal
to $r$.}
  \end{quote}
  \end{sltn}

\section{Existence Theorems}
\label{sec:existence_theorems}

A theorem that asserts the existence of something is called an \textbf{existence theorem}\index{existence theorem}. We prove existence theorems in two steps.
\begin{enumerate}
  \item Produce a "candidate". 
  \item We show that our candidate satisfies the theorem.
\end{enumerate}

Remember that when proving existence theorems, one does not actually need to
tell where the candidate came from. As long as it is given explicitly and is
shown to satisfies the requirements.

\section{Uniqueness Theorems}
\label{sec:uniqueness_theorems}

A theorem that guarantees the uniqueness of a mathematical object is called a
\textbf{uniqueness theorem}\index{uniqueness theorem}.
\begin{exmp}
  Assume that $x^3 + 37$ has a real root. Prove that it is has only one.
  \begin{proof}
    Assume that $x_1$ and $x_2$ are real numbers and that $x_1^3 + 37 = 0$ and
    $x_2^3 + 37 = 0$. Then $x_1 ^3 = x_2^3$. Since cube roots of real numbers
    are unique, $x_1 = x_2$.
  \end{proof}
\end{exmp}

\section{Examples and Counterexamples}
\label{sec:examples_and_counterexamples}

In order to prove that an implication is false, we need to provide a counterexample.
When we provide a counterexample, we are just showing that such a value exists. 

\begin{exrc}
  Give counterexamples to the following proposed (but false) statements.
  \begin{enumerate}
    \item If a real number is greater than 5, it is less than 10.
    \item If $x$ is a real number, $x^3 = x$.
    \item All prime numbers are odd numbers. \textit{What is the hypothesis
      here? What is the conclusion?}
    \item If $x+y$ is odd and $y+z$ is odd, then $x + z$ is odd.
    \item Given three distinct points in space, there is one and only one plane
      passing through all three points.
  \end{enumerate}
\end{exrc}
\begin{sltn}
  \begin{enumerate}
    \item Let $x$ be the real number $11$. Then $x > 5$ but $x \not< 10$.
    \item Let $x$ be the real number $2$. Then $x^3 = 8 \neq x$.
    \item The number $2$ is by definition a prime number, but it can be written
      on the form $2n$ where $n = 1$. Hence $2$ is an even prime number.  The
      hypothesis here is $x$ is a prime number, and the conclusion is $x$ is
      odd.
    \item Let $x = 1$, $y = 2$, and $z = 1$. Then $x + y$ is odd, $y + z$ is
      odd, but $x + z$ is even.
    \item If we chose three points in $ \mathbb{R}$ along the $x$-axis, say
      $(0, 0, 0), (1, 0, 0)$ and $(2, 0, 0)$ then we have an infinite number of
      planes passing through these three points.
  \end{enumerate}  
\end{sltn}

\begin{exrc}
  Think about the implications given in the thought experiment. When you
  decided that one of them was false, did you justify your conclusion by means
  of a counter example? 
\end{exrc}

\begin{sltn}
  In the cases where I immediately saw an appropriate counter-example, I
  described it.  I could probably rewrite the ones I did not find a proof for
  using the language we now have learned. 
\end{sltn}

The process in which we draw conclusions about the general based on the
particular is called \textbf{inductive reasoning}\index{inductive reasoning}.
This is contrasted with \textbf{deductive reasoning}\index{deductive
reasoning}, which is the process of using the rules of logic to deduce logical
consequences.  In other words, inductive reasoning is how we make conjectures
in mathematics, where as deductive reasoning is how we prove these conjectures.

The moral of the story is that providing a counterexample is \textbf{conclusive
proof}\index{conclusive proof} that an implication is false. Just checking a
large number of examples, without being able to check all of them does not
prove an implication in general.

\section{Direct Proof}
\label{sec:direct_proof}

Remember that the statement "If $A$, then $B$" is true for all values of the
variables that make $A$ true make $B$ true also. Thus, one way of verifying
that the implication holds is the check all possible cases where the hypothesis
holds and see if the conclusion is true.

When an implication is proven by assuming that the hypothesis is true and then
showing that the conclusion is also true, the proof is called a \textbf{direct
proof}\index{direct proof}.

An example of a direct proof is the proof for the fact that if $x$ is an even
integer, then $x^2$ is an even integer.

\begin{exrc}
  If you haven't done so already, use the method of direct proof to prove that
  if $x + y$ is an even integer and $y + z$ is an even integer, then $x + z$
  is an even integer.
\end{exrc}
\begin{sltn}
  We first assume that the hypothesis is true, and then we see if we can prove
  that the conclusion must be true as well. Assume that $x + y$ and $y + z$ are
  even numbers.  We then know that $x + y = 2n$ and $y + z = 2m$ for some
  appropriate integers $n$ and $m$.  Solving these two equations for $x$ and
  $z$ we get that $x = 2n - y$ and $z = 2m - y$. Hence 
  \begin{equation}
    \notag
    x + z = 2n - y + 2m - y = 2(n + m - y),
  \end{equation}
  where $n + m - y$ is an integer. Thus, $x + z$ is an even number. 
\end{sltn}

In addition to direct proof, we have two other common proof techniques, namely
proof by contrapositive and proof by contradiction.

\section{Proof by Contrapositive}
\label{sec:proof_by_contrapositive}


\printindex
\end{document}












